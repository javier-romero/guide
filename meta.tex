\section{Meta-Programming}\label{sec:meta}

This section is not yet ready for publishing
and will be included in one of the forthcoming editions of this guide.

Information on meta-programming can be obtained at the following references.

\begin{itemize}
\item Meta-programming relying on reified logic programs was introduced with \gringo~3;
  details can be found at:
  \begin{itemize}
  \item \cite{gekasc11b}
  \item \url{http://www.cs.uni-potsdam.de/wv/metasp}
  \end{itemize}

\item \gringo{} ships with a small tool called \reify\ to reify logic programs in \aspif{} format~\cite{kascwa17a}.

  Moreover, it offers a dedicated output format for reification, viz.\
  \lstinline{--output=reify}
  to output logic programs as sets of facts

  Also, \gringo~5 features options
  \lstinline{--reify-sccs}
  to add strongly connected components to reified output
  and
  \lstinline{--reify-steps}
  to add step numbers to reified output.
% ??? --rewrite-minimize    : Rewrite minimize constraints into rules

  An advantage of using the \reify{} tool is that it can be combined with \clasp's pre-processor.
\end{itemize}

%%% Local Variables: 
%%% mode: latex
%%% TeX-master: "guide"
%%% End: 
